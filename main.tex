\documentclass[12pt, dvipdfmx]{beamer}
\usepackage{graphicx,xcolor}
\usepackage{subfiles}
\usepackage{color}
\usepackage{tikz}
\usepackage{url}
\usetikzlibrary{calc,decorations.pathreplacing,quotes,positioning,shapes,fit,arrows,backgrounds,tikzmark}
\tikzset{Node/.style={circle, draw}}

\usetheme{metropolis}
\usefonttheme{professionalfonts}
\setbeamercolor{alerted text}{fg=mLightBrown!90!black}

\usepackage{bxdpx-beamer}
\usepackage{minijs}
\usepackage{latexsym}
\usepackage[deluxe,expert]{otf}
\renewcommand{\kanjifamilydefault}{\gtdefault}

\setbeamersize{text margin left=.75zw, text margin right=.75zw}
\setbeamertemplate{frame footer}{}

\usepackage{enumitem}
\setlist[itemize]{label=\textbullet}

\setbeamertemplate{section in toc}[ball unnumbered]
\setbeamertemplate{subsection in toc}[square]

\title{Rustはいいぞ}
\institute{北海道大学 情報知識ネットワーク研究室 M1}
\author{光吉 健汰}

\begin{document}
\maketitle
\begin{frame}{目次}
	\small
	\tableofcontents
\end{frame}

\begin{frame}{Rustとは?}
	効率的で信頼できるソフトウェアを誰もが作れる言語\footnotemark

	\begin{itemize}
		\item パフォーマンス C言語やC++言語と同等程度の処理速度
		\item 信頼性 null参照が(自ら踏み込まない限り)できない
		\item 生産性 ツールが最初から揃っている
	\end{itemize}

	最も愛されているプログラミング言語\footnotemark

	\footnotetext[1]{\url{https://www.rust-lang.org/ja}}
	\footnotetext[2]{\url{https://www.rust-lang.org/ja}}
\end{frame}

\begin{frame}{パフォーマンス}
	すごいはやい
\end{frame}

\begin{frame}{信頼性}
	所有権モデルすごい
\end{frame}

\begin{frame}{生産性}
	ツールがいっぱい
	\begin{itemize}
		\item Cargo ビルド、デバッグ、テスト、ドキュメント生成、パッケージ管理なんでもできる
		\item 各種エディタ拡張 VScode、ST3、IntelliJ、Vimなど様々なエディタ拡張を公式サポート
		\item Crate.io パッケージレジストリ、コミュニティが活発で多くのライブラリが公開されている
	\end{itemize}
\end{frame}

\end{document}
